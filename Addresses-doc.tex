%% Dr Alun Moon
%% assignment-template.tex, generated from assignment-template.dtx
\documentclass[12pt]{article}
\usepackage[british]{babel}

\usepackage{url}
\usepackage[round]{natbib}
\usepackage{infoboxes}

\usepackage{latexsym}
\usepackage{tlatex}
\usepackage{color}
\definecolor{boxshade}{gray}{.8}
\setboolean{shading}{true}

\usepackage{enumerate,
            siunitx}

\begin{document}

\section{Address Book --- v1}
\subsection{The Specification} 
\begin{tla}
---------------------------- MODULE addressbook ----------------------------
CONSTANTS Names, Emails
VARIABLE address

Invariant == address \in [Names -> Emails ]

Init == address = <<>> \* A different ideom for an empty function...

Add(name,email) == 
    address' = [n \in DOMAIN address \union {name} |->
                         IF n \in DOMAIN address 
                         THEN address[n]
                         ELSE email
               ]

Remove(name) == 
    address' = [ n \in DOMAIN address \ {name} |-> address[n] ]

Find(name) ==  
    /\ name \in DOMAIN address
    /\ UNCHANGED address

Next == 
    \E n \in Names , e \in Emails : 
        \/ Add(n,e)
        \/ Remove(n)
        \/ Find(n)

=============================================================================
\* Modification History
\* Last modified Sat Feb 13 10:50:44 GMT 2021 by alunm
\* Created Wed Feb 10 21:49:46 GMT 2021 by alunm
\end{tla}
\begin{tlatex}
\@x{}\moduleLeftDash\@xx{ {\MODULE} addressbook}\moduleRightDash\@xx{}%
\@x{ {\CONSTANTS} Names ,\, Emails}%
\@x{ {\VARIABLE} address}%
\@pvspace{8.0pt}%
\@x{ Invariant \.{\defeq} address \.{\in} [ Names \.{\rightarrow} Emails ]}%
\@pvspace{8.0pt}%
\@x{ Init \.{\defeq} address \.{=} {\langle} {\rangle}}%
\@y{%
  A different ideom for an empty function...
}%
\@xx{}%
\@pvspace{8.0pt}%
\@x{ Add ( name ,\, email ) \.{\defeq}}%
 \@x{\@s{25.93} address \.{'} \.{=} [ n \.{\in} {\DOMAIN} address \.{\cup} \{
 name \} \.{\mapsto}}%
\@x{\@s{121.75} {\IF} n \.{\in} {\DOMAIN} address}%
\@x{\@s{121.75} \.{\THEN} address [ n ]}%
\@x{\@s{121.75} \.{\ELSE} email}%
\@x{\@s{82.88} ]}%
\@pvspace{8.0pt}%
\@x{ Remove ( name ) \.{\defeq}}%
 \@x{\@s{16.4} address \.{'} \.{=} [ n \.{\in} {\DOMAIN} address
 \.{\,\backslash\,} \{ name \} \.{\mapsto} address [ n ] ]}%
\@pvspace{8.0pt}%
\@x{ Find ( name ) \.{\defeq}}%
\@x{\@s{25.10} \.{\land} name \.{\in} {\DOMAIN} address}%
\@x{\@s{25.10} \.{\land} {\UNCHANGED} address}%
\@pvspace{8.0pt}%
\@x{ Next \.{\defeq}}%
\@x{\@s{16.4} \E\, n \.{\in} Names ,\, e \.{\in} Emails \.{:}}%
\@x{\@s{29.16} \.{\lor} Add ( n ,\, e )}%
\@x{\@s{29.16} \.{\lor} Remove ( n )}%
\@x{\@s{29.16} \.{\lor} Find ( n )}%
\@pvspace{8.0pt}%
\@x{}\bottombar\@xx{}%
\@x{}%
\@y{%
  Modification History
}%
\@xx{}%
\@x{}%
\@y{%
  Last modified Sat Feb 13 10:50:44 GMT 2021 by alunm
}%
\@xx{}%
\@x{}%
\@y{%
  Created Wed Feb 10 21:49:46 GMT 2021 by alunm
}%
\@xx{}%
\end{tlatex}

\section{The Model}

\subsection{Model Overview}
\paragraph{The Behaviour specification} is an \emph{Initial-predicate and
next-state relation} 
\begin{description}
	\item[Initial Predicate] \textit{Init}
	\item[Next-state relation] \textit{Next}
\end{description}
\paragraph{The Model} values assigned to declared constants
\begin{description}
	\item[Set of Names] is set to
\begin{tla}
Names <- { "a", "b" }
\end{tla}
\begin{tlatex}
\@x{ Names \.{\leftarrow} \{\@w{a} ,\,\@w{b} \}}%
\end{tlatex}
\item[Set of email addresses] is set to
\begin{tla}
Emails <- {"1","2"}
\end{tla}
\begin{tlatex}
\@x{ Emails \.{\leftarrow} \{\@w{1} ,\,\@w{2} \}}%
\end{tlatex}
\end{description}

\subsection{Checks and verifications}
No invariants were checked.

\subsection{Results} A summary of the results
\paragraph{Statistics} a summaries of the actions and number of states
found.

\begin{table}[h]
\begin{tabular}{lr}
	States found & \num{97} \\
 Distinct states & \num{9} \\ 
\end{tabular}
\end{table}

\begin{table}[h]
\begin{tabular}{llrr}
	\textbf{Action} & Location & States Found & \textbf{Distinct states} \\
	\hline
	\textit{Init}   & Line 7 & 1 & 1 \\
	\textit{Add}    & Line 9 & 36 & 6 \\
	\textit{Remove} & Line 16 & 36 & 2 \\
	\textit{Find}   & Line 19 & 24 & 0 \\
\end{tabular}
\end{table}

\subsection{Discussion}
\subsubsection{Model description} 
\paragraph{The state of the system is .} A function that maps from names to
addresses.

\subparagraph{The initial conditions} are an empty function, which can be also
written as an empty sequence.

\paragraph{The Next relation} is that a name and address can be added or
removed from the function, or an address found.

\subsubsection{Comments}
Using a function we run into some of it's limits.  With the invariant stated,
the domain is a set of names; an initial value of an empty function does not
have its domain in the set described by the invariant.

\begin{infobox}{\tricky}
Using a function to model the look-up does work and is the right thing.
The problem with this case is that the initial condition of an empty function
is not stricktly in the set of functions stated in the invariant.
The problem here is stating the right invariant.
\end{infobox}

\subsubsection{Interpretation of results}
The number of distict states found for the \textit{Remove} operation is
initially surprising, however it is correct, the state of the system after an
entry has been removed is the same as the state before the entry was added.

\clearpage

\section{Address book --- v2}
\subsection{The Specification} 
\begin{tla}
---------------------------- MODULE addressbook ----------------------------
CONSTANTS Names, Emails
VARIABLE address

Invariant == address \in SUBSET [name:Names, email:Emails ]

Init == address = {}

Find(name) ==  
     \E a \in address : a.name = name


Add(name,email) == 
    /\ \lnot Find(name)  \* not in address book
    /\ address' = address \union { [name|-> name, email|->email] }

Remove(name) == 
    /\ Find(name)
    /\ address' = address \ {CHOOSE a \in address : a.name = name }

Next == 
    \E n \in Names , e \in Emails : 
        \/ Add(n,e)
        \/ Remove(n)
        \/ Find(n)/\UNCHANGED address

=============================================================================
\* Modification History
\* Last modified Sat Feb 13 11:31:50 GMT 2021 by alunm
\* Created Wed Feb 10 21:49:46 GMT 2021 by alunm
\end{tla}
\begin{tlatex}
\@x{}\moduleLeftDash\@xx{ {\MODULE} addressbook}\moduleRightDash\@xx{}%
\@x{ {\CONSTANTS} Names ,\, Emails}%
\@x{ {\VARIABLE} address}%
\@pvspace{8.0pt}%
 \@x{ Invariant \.{\defeq} address \.{\in} {\SUBSET} [ name \.{:} Names ,\,
 email \.{:} Emails ]}%
\@pvspace{8.0pt}%
\@x{ Init \.{\defeq} address \.{=} \{ \}}%
\@pvspace{8.0pt}%
\@x{ Find ( name ) \.{\defeq}}%
\@x{\@s{29.67} \E\, a \.{\in} address \.{:} a . name \.{=} name}%
\@pvspace{16.0pt}%
\@x{ Add ( name ,\, email ) \.{\defeq}}%
\@x{\@s{25.93} \.{\land} {\lnot} Find ( name )\@s{4.1}}%
\@y{%
  not in address book
}%
\@xx{}%
 \@x{\@s{25.93} \.{\land} address \.{'} \.{=} address \.{\cup} \{ [ name
 \.{\mapsto} name ,\, email \.{\mapsto} email ] \}}%
\@pvspace{8.0pt}%
\@x{ Remove ( name ) \.{\defeq}}%
\@x{\@s{16.4} \.{\land} Find ( name )}%
 \@x{\@s{16.4} \.{\land} address \.{'} \.{=} address \.{\,\backslash\,} \{
 {\CHOOSE} a \.{\in} address \.{:} a . name \.{=} name \}}%
\@pvspace{8.0pt}%
\@x{ Next \.{\defeq}}%
\@x{\@s{16.4} \E\, n \.{\in} Names ,\, e \.{\in} Emails \.{:}}%
\@x{\@s{29.16} \.{\lor} Add ( n ,\, e )}%
\@x{\@s{29.16} \.{\lor} Remove ( n )}%
\@x{\@s{29.16} \.{\lor} Find ( n ) \.{\land} {\UNCHANGED} address}%
\@pvspace{8.0pt}%
\@x{}\bottombar\@xx{}%
\@x{}%
\@y{%
  Modification History
}%
\@xx{}%
\@x{}%
\@y{%
  Last modified Sat Feb 13 11:31:50 GMT 2021 by alunm
}%
\@xx{}%
\@x{}%
\@y{%
  Created Wed Feb 10 21:49:46 GMT 2021 by alunm
}%
\@xx{}%
\end{tlatex}

\section{The Model}


\subsection{Model Overview}
The same model was used as for version 1

\subsection{Checks and verifications}
\paragraph{Invariants} We now have an invariant that can be checked.
\textit{Invariant}

\subsection{Results} A summary of the results
\paragraph{Statistics} a summaries of the actions and number of states
found.

\begin{table}[h]
\begin{tabular}{lr}
	States found & \num{61} \\
 Distinct states & \num{9} \\ 
\end{tabular}
\end{table}

\begin{table}[h]
\begin{tabular}{llrr}
	\textbf{Action} & Location & States Found & \textbf{Distinct states} \\
	\hline
	\textit{Init}   & Line 7 & 1 & 1 \\
	\textit{Add}    & Line 13 & 12 & 8 \\
	\textit{Remove} & Line 17 & 24 & 0 \\
	\textit{Next}   & Line 25 & 24 & 0 \\
\end{tabular}
\end{table}

\subsection{Discussion}
\subsubsection{Model description} 
\paragraph{The state of the system is .} now modelled as a set of records,
which an reflection seems a good description of a database.
The initial conditions are now just a simple empty set.
The invariant is now a subset of all possible records.

The Find operation is only marginally more complex.  The operation has to be a
predicate to count as an action that can be performed.  It tests \emph{if} an
entry is found in the set.  It is also useful as a guard condition on the add
and remove operations.

\begin{infobox}{\info}
This illustrates how there are more than one way of specifying a system.  But
as how both can satisfy the model and requirements \emph{both} are correct
specifications.
\end{infobox}

\end{document}

\endinput
%%
%% End of file `assignment-template.tex'.
